\documentclass[9pt]{article} 

\usepackage[utf8]{inputenc}
\usepackage{geometry}
\usepackage{amsmath}
\usepackage{mathtools}
\usepackage{fixltx2e}
\usepackage{url}
\usepackage{graphicx}
\usepackage{mathabx}
\usepackage{float}
\usepackage{subfig}
\usepackage{caption}
\usepackage{picture}
\usepackage{verbatim}
\usepackage{pdfpages} 
\usepackage{longtable}
\usepackage{tabu}
\usepackage{hyperref}

\geometry{a4paper} 
\geometry{margin=3cm} 

\title{Software Project Management Plan\\Version 0.3}
\author{Lars Van Holsbeeke\\Software Engineering 2013-2014 Group 1}

\begin{document}

\begin{titlepage}
\maketitle

Revision History

\begin{longtable}[c]{@{}lll@{}}
\hline\noalign{\medskip}
Version & Date & Description
\\\noalign{\medskip}
\hline\noalign{\medskip}
\textbf{0.1} & 29/10/2013 & Creation of document structure
\\\noalign{\medskip}
\textbf{0.2} & 03/10/2013 & Completion of initial version
\\\noalign{\medskip}
\textbf{0.3} & 14/11/2013 & Adapted to feedback of initial version
\\\noalign{\medskip}
\hline
\end{longtable}

\end{titlepage}

\tableofcontents
\clearpage

\section{Overview}\label{overview}

\subsection{Project Summary}\label{project-summary}

\subsubsection{Purpose, scope, and
objectives}\label{purpose-scope-and-objectives}

The main purpose of this project is to create a working scheduling
webapplication with specific support for mobile devices like smartphones
and tablets that enables (authorized) users to query their personal
course/final schedule and notifies them about last-minute changes. We
will call this application: \texttt{Xiast}\emph{(\textbf{X}iast
\textbf{i}s \textbf{a} \textbf{s}cheduling \textbf{t}ool)} More specific
requirements can be found in the \hyperref[SRS]{SRS} (Software
Requirements Specification) document.

The main system itself uses the \texttt{Wilma} server of the university
as back-end and a normal or mobile browser as front-end.

All documents, source code and other artifacts are publicly available on
Github. Documents can be found under
\href{https://github.com/se1-1314/xiast-docs}{xiast-docs}, source code
can be found under \href{https://github.com/se1-1314/xiast}{xiast}.

This academic 3rth bachelor project is part of the course
``\hyperref[SoftEng]{Software Engineering}'', taught by dr. R. Van Der
Straeten taking place at the ``Vrije Universiteit Brussel''

\subsubsection{Assumptions and
constraints}\label{assumptions-and-constraints}

Some constraints involving documentation standards, infrastructure and
use of certain technologies have been defined by the client:

\paragraph{Documentation}\label{documentation}

\begin{itemize}
\itemsep1pt\parskip0pt\parsep0pt
\item
  This document (SPMP) must conform the IEEE 1058-1998 standard
\item
  The SRD, SDD, STD, SQAP and SCMP must also conform their IEEE xxx-1998
  standard or a more recent revision of that standard.
\item
  All documents must be written or in Dutch or in English, but not a
  combination of the two.
\item
  All documents must be available in the PDF format
\item
  At least following documents must be maintained:

  \begin{itemize}
  \itemsep1pt\parskip0pt\parsep0pt
  \item
    Software Project Management Plan (SPMP)
  \item
    Software Test Plan (STD)
  \item
    Software Requirements Specification (SRS)
  \item
    Software Design Document (SDD)
  \end{itemize}
\item
  Meeting minutes must be made for all meetings
\item
  An SCMP and an SQMP are not necessary, but all relevant information
  concerning them must be found in the SPMP.
\end{itemize}

\paragraph{Language}\label{language}

\begin{itemize}
\itemsep1pt\parskip0pt\parsep0pt
\item
  Only Java, JavaScript, HTML, CSS, SQL and corresponding libraries and
  open-source frameworks
\item
  Only open-source software may be used for both the endproduct and
  tools
\item
  A particular choice of library, tool, etc. must be motivated by means
  of reliabilityn, openness and simplicity.
\item
  A library can only be used after agreement with the client and a
  comparative study of other possible libraries.
\end{itemize}

\paragraph{Infrastucture}\label{infrastucture}

\begin{itemize}
\itemsep1pt\parskip0pt\parsep0pt
\item
  The VUB ``\hyperref[Wilma]{Wilma}'' server must be used as backend for
  the system.
\item
  The system must work on a browser as frontend.
\item
  The system must work on a mobile browser.
\end{itemize}

\paragraph{Other Constraints}\label{other-constraints}

\begin{itemize}
\itemsep1pt\parskip0pt\parsep0pt
\item
  ``\hyperref[Github]{Github}'' must be used as public repository for
  the code.
\item
  All documents, source code and other artefacts must be publicly
  available in a structured way.
\item
  The system must have a standard, easy installation procedure.
\item
  The UI must be simple and attractive to use.
\item
  Requirements IDs may never be renumbered.
\item
  All of the code needs to be documentated.
\item
  Test must be written using the ``\hyperref[JUnit]{JUnit}'' framework.
\item
  The system must be modular in design to accomodate extension and
  replacement of the containing modules.
\item
  The development proces must be iterative with incremental delivery.
\end{itemize}

\subsubsection{Project deliverables}\label{project-deliverables}

The table table below shows code, document and other deliverables with
their corresponding deadline: 9 o'clock in the morning on the date
shown.

\begin{longtable}[c]{@{}ll@{}}
\hline\noalign{\medskip}
Date & Deliverable
\\\noalign{\medskip}
\hline\noalign{\medskip}
04/11/2013 & First version of the SPMP
\\\noalign{\medskip}
15/11/2013 & First version of documents
\\\noalign{\medskip}
18/11/2013 & Data dump: data available for use
\\\noalign{\medskip}
13/12/2013 & End of first iteration: delivery of code and documents
\\\noalign{\medskip}
18/12/2013 & First presentation
\\\noalign{\medskip}
04/03/2014 & End of second iteration: delivery of code and documents
\\\noalign{\medskip}
12/03/2014 & Second presentation
\\\noalign{\medskip}
15/04/2014 & End of thirth iteration: delivery of code and documents
\\\noalign{\medskip}
16/05/2014 & End of fourth iteration: final delivery of code and
documents
\\\noalign{\medskip}
21/05/2014 & Final presentation
\\\noalign{\medskip}
\hline
\end{longtable}

\subsubsection{Schedule}\label{schedule}

Section 5.2 describes the work plan of the project, which contains a
detailed description of the work activities with the corresponding
teammembers that work on it along with an estimation of time they will
need to complete it.

\subsection{Evolution of the SPMP}\label{evolution-of-the-spmp}

This SPMP will be reviewed at least one time a week by the
projectmanager. If needed, this document will be updated by the same
person. Each (major) update will be logged in the
\hyperref[RevisionHistory]{Revision History}, to be found at the
beginning of this document.

\section{References}\label{references}

\begin{enumerate}
\def\labelenumi{\arabic{enumi}.}
\item
  SRS: Software Requirements Specifiaction

  Anders Deliens\\
  https://github.com/se1-1314/xiast-docs/blob/master/management/requirements/requirements.md
\item
  Software Engineering course, VUB

  Catalog number: 1004483BNR\\
  https://caliweb.cumulus.vub.ac.be/caliweb/?page=course-\\offer\&id=001462\&anchor=2\&target=pr\&year=1314\&language=en\&output=html
\item
  Wilma backend server

  http://wilma.vub.ac.be
\item
  Github

  https://github.com
\item
  JUnit framework

  http://junit.org/
\end{enumerate}

\begin{enumerate}
\def\labelenumi{\arabic{enumi}.}
\setcounter{enumi}{8}
\item
  Markable.in

  Online document writing tool for the Markdown language.
  \url{markable.in}
\item
  Iterative and Incremental development model

  Is any combination of both iterative design or iterative method and
  incremental build model for development. For more information:\\
  http://en.wikipedia.org/wiki/Iterative\_and\_incremental\_development
\item
  Agile Software Development

  Topic in the Software Engineering course. Is a group of software
  development methods based on \hyperref[IterativeIncremental]{iterative
  and incremental development}, where requirements and solutions evolve
  through collaboration between self-organizing, cross-functional teams.
  More information on the course slides or
  http://en.wikipedia.org/wiki/Agile\_software\_development
\item
  Boehm's spiral model

  Is a risk-driven process model generator for software projects.\\
  Further information: http://en.wikipedia.org/wiki/Spiral\_model
\end{enumerate}
\clearpage
\section{Definitions}\label{definitions}

\begin{longtable}[c]{@{}ll@{}}
\hline\noalign{\medskip}
Acronym & Declaration
\\\noalign{\medskip}
\hline\noalign{\medskip}
\textbf{DaM} & Database Manager
\\\noalign{\medskip}
\textbf{DeM} & Design Manager
\\\noalign{\medskip}
\textbf{CM} & Configuration Manager
\\\noalign{\medskip}
\textbf{IEEE} & Institute of Electrical and Electronics Engineers
\\\noalign{\medskip}
\textbf{PM} & Project Manager
\\\noalign{\medskip}
\textbf{RM} & Requirements Manager
\\\noalign{\medskip}
\textbf{QAM} & Quality Assurance Manager
\\\noalign{\medskip}
\textbf{SDD} & Software Design Document
\\\noalign{\medskip}
\textbf{SPMP} & Software Project Magement Plan
\\\noalign{\medskip}
\textbf{SRS} & Software Requirements Specification
\\\noalign{\medskip}
\textbf{STD/STP} & Software Test Plan
\\\noalign{\medskip}
\textbf{SQAP} & Software Quality Assurance Plan
\\\noalign{\medskip}
\textbf{SDP} & Software Documentation Plan
\\\noalign{\medskip}
\textbf{VUB} & Vrije Universiteit Brussel
\\\noalign{\medskip}
\textbf{PDF} & Portable Document Format
\\\noalign{\medskip}
\textbf{UI} & User Interface
\\\noalign{\medskip}
IDE & Integrated Development Environment
\\\noalign{\medskip}
\hline
\end{longtable}

Other definitions can be found on page 2-3 of the IEEE 1058-1998
standard for Software Project Management Plans

\section{Project Organisation}\label{project-organisation}

\subsection{External interfaces}\label{external-interfaces}

Client

In this project the titular of this course, \hyperref[SoftEng]{Software
Engineering}, mrs. R. Van Der Straeten, will together with her
assistant, mr. J Nicolay, act as client for the project. This means that
all communication involving requirements and design will pass by at
least one of them and respectively the
\hyperref[RequirementsManager]{Requirements Manager} and the
\hyperref[Designleader]{Design Leader}. All other communication with the
client will be handled by the \hyperref[Projectmanager]{Projectmanager},
this includes submitting deliverables: source-code and documents,
communication involving presentations, etc.

\subsubsection{Infrastructure}\label{infrastructure}

All communication concerning the available infrastructure: the
\hyperref[Wilma]{Wilma} backend server will be handled with the head of
infrastructure, mr. D. Van Deun by the web- and databasemanager.

\subsubsection{External Scheduling Data}\label{external-scheduling-data}

Any problems, remarks,\ldots{} involving the dump of scheduling data on
November 18th, 2013 will be communicated to the infrastructure manager,
mr. D. Van Deun.

\subsection{Internal Structure}\label{internal-structure}

\subsubsection{Internal Communication}\label{internal-communication}

All communication between the teammembers outside meetings must be
logged by or the issue tracker on \hyperref[Github]{Github} or using the
internal mailinglist: se1\_1314@wilma.vub.ac.be. This is a rule of thum
that must be followed by the teammembers. Only if the information to
communicate is such unimportant, irrelevant to the other teammembers,
does not involve agreements, deadlines, etc. and the urgency of the
concerning activities is very low, teammembers can use private mail. In
case of urgent problems, problems with another teammember, important
matters that need immediate attention, etc. teammembers may use the
private mobile phone number of the
\hyperref[Projectmanager]{Projectmanager} that has been given to them in
the second meeting.

\subsubsection{Internal Organisation}\label{internal-organisation}

The chart below shows the internal organisation and flows of information
between the actors of the team:

The \hyperref[Projectmanager]{Projectmanager} acts as a central
coordination point for the whole team, he also communicates with the
client (see \hyperref[4.1.1]{Client}). Communicationflows with the
client are colored red.

\subsection{Roles and
responsibilities}\label{roles-and-responsibilities}

\begin{itemize}
\itemsep1pt\parskip0pt\parsep0pt
\item
  Project Manager

  \begin{itemize}
  \itemsep1pt\parskip0pt\parsep0pt
  \item
    Creating \& providing the SPMP with updates
  \item
    Coordination of the team
  \item
    Contact person for all teammembers
  \item
    Chairman during meetings
  \item
    Creating a weekly meeting agenda on Github
  \item
    Approving decisions taken during meetings
  \item
    Detecting team related problems and solving them
  \item
    Ensuring deadlines are met by all teammembers
  \item
    Ensuring quality of non-code artefacts, created by the teammembers
  \item
    Verifying (together with the secretary) meeting minutes and
    correcting them if needed
  \item
    Creation of a time-scheme, together with the other teammembers
  \item
    Creation of annotated tags on the Github repository: one for each
    iteration
  \end{itemize}
\item
  Configuration Manager

  \begin{itemize}
  \itemsep1pt\parskip0pt\parsep0pt
  \item
    Creating \& providing the SCMP with updates
  \item
    Managing the \hyperref[Github]{Github} repository for code and
    documents
  \item
    Managing tools used within the team
  \item
    Providing some documentation concerning the used tools and Git.
  \item
    Ensuring safety and restorability of documents
  \end{itemize}
\item
  Quality Assurance Leader

  \begin{itemize}
  \itemsep1pt\parskip0pt\parsep0pt
  \item
    Creation of \& providing the STP with updates
  \item
    Optionally creating (and maintaining) an SQAP
  \item
    Quality-based Monitoring of the Software
  \item
    Reviewing source-code: are all required features implemented?
  \item
    Setting up \hyperref[JUnit]{JUnit} tests
  \end{itemize}
\item
  Requirements Management Leader

  \begin{itemize}
  \itemsep1pt\parskip0pt\parsep0pt
  \item
    Creation of \& providing the SRS with updates
  \item
    Communicating with client about requirements: p.e. in case of
    ambiguity, special requests, etc.
  \item
    Determines the priority for each working activity
  \item
    Takes care that activities with higher priority are done first
  \item
    Reporting possible changes to the requirements, made by the client
  \end{itemize}
\item
  Design Leader

  \begin{itemize}
  \itemsep1pt\parskip0pt\parsep0pt
  \item
    Creation of \& providing the SDD with updates
  \item
    Determining (and managing) the architecture of the system and
    Database
  \item
    Communicating with the client about the design
  \end{itemize}
\item
  Implementation Leader

  \begin{itemize}
  \itemsep1pt\parskip0pt\parsep0pt
  \item
    Managing of the source code
  \item
    Reporting issues concerning the source code on meetings
  \item
    Distributing programming workload to all teammembers
  \item
    Monitoring developers
  \end{itemize}
\item
  Server Manager

  \begin{itemize}
  \itemsep1pt\parskip0pt\parsep0pt
  \item
    Regularly updates the website with new information
  \item
    Takes care of communication with the infrastructure manager
  \item
    Manages database, server applications and related services
  \end{itemize}
\end{itemize}

\section{Managerial Process Plans}\label{managerial-process-plans}

\subsection{Start-up Plan}\label{start-up-plan}

\subsubsection{5.1.2 Staffing Plan}\label{staffing-plan}

 H = Function Holder, B = Back-up

%\begin{longtable}[c]{@{}lcccccc@{}}
\begin{longtabu}[c]{@{}lXXXXXX@{}}
\hline\noalign{\medskip}
Function/Teammember & Youssef Boudiba & Anders Deliens & Adriaan Leijnse
& Kwinten Pardon & Nils Van Geele & Lars Van Holsbeeke
\\\noalign{\medskip}
\hline\noalign{\medskip}
Project Manager & & B & & & & H
\\\noalign{\medskip}
Configuration Manager & & & & & H & B
\\\noalign{\medskip}
Quality Assurance Leader & H & & & B &
\\\noalign{\medskip}
Requirements Manager & & H & B & &
\\\noalign{\medskip}
Design Leader & & & H & & B
\\\noalign{\medskip}
Implementation Leader & B & & & H &
\\\noalign{\medskip}
Secretary & H & B & & &
\\\noalign{\medskip}
Server Manager & & & B & & H
\\\noalign{\medskip}
\hline
\end{longtabu}
%\end{longtable}

\subsection{Work Plan}\label{work-plan}

\subsubsection{Work activities}\label{work-activities}

The table below shows an overview of the different activities in the
development process together with the responsible teammember and an
estimation of time needed to complete the activity. The estimated time
may differ from the actual performed time

\begin{longtable}[c]{@{}llrl@{}}
\hline\noalign{\medskip}
Activity & Responsible & Estimated Time & Documents
\\\noalign{\medskip}
\hline\noalign{\medskip}
Team management & PM & & SPMP
\\\noalign{\medskip}
Configuration management & CM & & SCMP
\\\noalign{\medskip}
Quality Checks & QAM & & n.a.
\\\noalign{\medskip}
Requirements management & RM & & SRS
\\\noalign{\medskip}
Design & DeM & & SDD
\\\noalign{\medskip}
Tests & QAM & & STP
\\\noalign{\medskip}
Implementation & IL, programmers & & source code
\\\noalign{\medskip}
\hline
\end{longtable}

Please note that an estimation of time is not yet made in this version
of the SPMP. One reason for this is lack of experience. Nevertheless
will this estimation be made at the next teammeeting.

During the development proces, each teammember will log how much time he
spends on an activity of the project. This includes time spend on
programming, documentation, testing, versioning control, etc. but also
time spend on meetings. At every (weekly) meeting, each team member
should tell how much time he has spent on which activity, with a clear
separation between managing and coding.

\subsubsection{Schedule allocation}\label{schedule-allocation}

A GANTT chart will be used for this. It will be made at the next
teammeeting when a License for Microsoft Project 2013 has been obtained.

\subsubsection{Resource allocation}\label{resource-allocation}

An overview of rescources that will be used can be found in the table
below

\begin{longtable}[c]{@{}ll@{}}
\hline\noalign{\medskip}
Rescource & Activities
\\\noalign{\medskip}
\hline\noalign{\medskip}
\hyperref[Wilma]{Wilma} backend server & Application backend; Hosting of
the static website
\\\noalign{\medskip}
Microsoft Project & Project Management (tool)
\\\noalign{\medskip}
Microsoft PowerPoint & Presentations
\\\noalign{\medskip}
\hyperref[markable]{Markable} & Writing documents in the Markdown
language
\\\noalign{\medskip}
Github & Versioning Control System
\\\noalign{\medskip}
Smartphone (Android) & Testing mobile version of the tool
\\\noalign{\medskip}
\hline
\end{longtable}

\subsection{Control Plan}\label{control-plan}

\subsubsection{Requirements control
plan}\label{requirements-control-plan}

Possible changes of requirements will always be communicated between the
requirements manager and the client. When a change occurs, the
requirements manager puts an new topic on the agenda of the next
teammeeting and updates the SRS.

\subsubsection{Schedule control plan}\label{schedule-control-plan}

Problems involving scheduling, deadlines, etc. will be discussed during
the weekly meeting. Each teammember is responsible to keep track of his
deadlines, and will report (at the weekly meeting) what he has done on
which activity during the last week. The projectmanager himself will
keep track of the global planning by using these reports and make
adjustments to the planning and/or activity if needed. If it seems that
one of the teammembers won't make the deadline, one or more other
teammembers can jump in on the activity concerned. This is highly
appreciated.

\subsubsection{Quality control plan}\label{quality-control-plan}

All code and documentation will be periodically checked by the Quality
Assurance Manager and before the end of each iteration. First he reports
(if needed) to the concerning person. If any severe (quality based)
problems are detected, he will report also them at the weekly meeting.

\subsubsection{Reporting plan}\label{reporting-plan}

Using the SPMP, SCMP, STD and SDD, the status of the project will be
reported to external entities (p.e. the client). All this documents are
free to be read by anybody on our \hyperref[Github]{Github} repository.
It can be reached and downloaded by using our
\href{http://wilma.vub.ac.be/~se1_1314}{static website} on
http://wilma.vub.ac.be/\textasciitilde{}se1\_1314

\subsection{Risk management plan}\label{risk-management-plan}

This list will be extended in future versions of this document All
estimations are on a scale from 0 to 10.

\begin{enumerate}
\def\labelenumi{\arabic{enumi}.}
\itemsep1pt\parskip0pt\parsep0pt
\item
  One of the teammembers is sick or leaves

  \begin{itemize}
  \itemsep1pt\parskip0pt\parsep0pt
  \item
    Probability: 3
  \item
    Impact: 6
  \item
    Priority: 9
  \item
    Cost of solution: 8
  \item
    Solution: Teammember with corresponding back-up function takes over.
  \item
    Target completion date: n.a.
  \item
    Responsible: Project Manager
  \end{itemize}
\item
  Bad communication between teammembers

  \begin{itemize}
  \itemsep1pt\parskip0pt\parsep0pt
  \item
    Probability: 5
  \item
    Impact: 6
  \item
    Priority: 8
  \item
    Cost of solution: 4
  \item
    Solution: Don't use too much private communication, use the
    mailinglist. The issue tracker on \hyperref[Github]{Github} must be
    up-to-date at all times.
  \item
    Target completion date: n.a.
  \item
    Responsible: Project Manager
  \end{itemize}
\item
  Not meeting deadlines

  \begin{itemize}
  \itemsep1pt\parskip0pt\parsep0pt
  \item
    Probability: 5
  \item
    Impact: 10
  \item
    Priority: 6
  \item
    Cost of solution: 5
  \item
    Solution: Keeping track of progress made using
    \hyperref[Github]{Github} functionality, weekly progress reports of
    teammembers.
  \item
    Target completion date: n.a.
  \item
    Responsible: Project Manager
  \end{itemize}
\item
  Lack of software quality

  \begin{itemize}
  \itemsep1pt\parskip0pt\parsep0pt
  \item
    Probability: 3
  \item
    Impact: 3
  \item
    Priority: 2
  \item
    Cost of solution: 5
  \item
    Solution: Periodically quality checks, tests,\ldots{} Reporting them
    to the weekly meeting. QAM gives recommendations to the teammembers
    on the weekly meeting and by using the mailing list. Making and
    resolving issues on the \hyperref[Github]{Github} issue tracker.
  \item
    Target completion date: n.a.
  \item
    Responsible: Quality Assurance Manager
  \end{itemize}
\item
  Misunderstandings between client and team

  \begin{itemize}
  \itemsep1pt\parskip0pt\parsep0pt
  \item
    Probability: 3
  \item
    Impact: 8
  \item
    Priority: 8
  \item
    Cost of solution: 5
  \item
    Solution: Regular meetings with the client to check if product meets
    expectations
  \item
    Target completion date: n.a.
  \item
    Responsible: Requirements Manager
  \end{itemize}
\item
  Not enough knowledge concerning the used programming language (p.e.
  Clojure, JavaScript, HTML5,\ldots{})

  \begin{itemize}
  \itemsep1pt\parskip0pt\parsep0pt
  \item
    Probability: 10
  \item
    Impact: 8
  \item
    Priority: 7
  \item
    Cost of solution: 7
  \item
    Solution: Watching tutorials, asking teammembers that know the
    language for help
  \item
    Target completion date: end of 1st iteration (Clojure basics), end
    of 2nd iteration (JavaScript, HTML5)
  \item
    Responsible: Design Manager, Implementation Leader
  \end{itemize}
\item
  Back-end server goes (temporarly) down

  \begin{itemize}
  \itemsep1pt\parskip0pt\parsep0pt
  \item
    Probability: 2
  \item
    Impact: 6
  \item
    Priority: 7
  \item
    Cost of solution: 2
  \item
    Solution: Using a mirror server: Aphrodite
  \item
    Target completion date: n.a.
  \item
    Responsible: Infrastructure Manager, Configuration Manager
  \end{itemize}
\item
  Github Versioning Control System goes down

  \begin{itemize}
  \itemsep1pt\parskip0pt\parsep0pt
  \item
    Probability: 1
  \item
    Impact: 6
  \item
    Priority: 7
  \item
    Cost of solution: 2
  \item
    Solution: Using the backup server (Aphrodite) running Gitlab
  \item
    Target completion date: n.a.
  \item
    Responsible: Configuration Manager
  \end{itemize}
\item
  Conflicts between teammembers

  \begin{itemize}
  \itemsep1pt\parskip0pt\parsep0pt
  \item
    Probability: 3
  \item
    Impact: 8
  \item
    Priority: 8
  \item
    Cost of solution: 7
  \item
    Solution: Negotiation between the teammembers involved together with
    the projectmanager.
  \item
    Target completion date: n.a.
  \item
    Responsible: Project Manager
  \end{itemize}
\item
  Abrupt changes in requirements

  \begin{itemize}
  \itemsep1pt\parskip0pt\parsep0pt
  \item
    Probability: 2
  \item
    Impact: 7
  \item
    Priority: 8
  \item
    Cost of solution: 6
  \item
    Solution: Using the modularity of the software product to implement
    as easily and efficiently as possible the changes. Prevention by
    involving the client in the development process.
  \item
    Target completion date: n.a.
  \item
    Responsible: Requirements manager, Implementation Leader
  \end{itemize}
\item
  Client cancels the project

  \begin{itemize}
  \itemsep1pt\parskip0pt\parsep0pt
  \item
    Probability: 1
  \item
    Impact: 10
  \item
    Priority: 10
  \item
    Cost of Solution: 10
  \item
    Solution: Closing down the project after double checking/negotiating
    with the client
  \item
    Target completion date: n.a.
  \item
    Responsible: Project Manager
  \end{itemize}
\item
  Wrong interpretation of requirements by the team

  \begin{itemize}
  \itemsep1pt\parskip0pt\parsep0pt
  \item
    Probability: 5
  \item
    Impact: 7
  \item
    Priority: 8
  \item
    Cost of Solution: 6
  \item
    Solution: Using the modularity of the software product to correct as
    easily and efficiently as possible the requirements that were
    misunderstood
  \item
    Target completion date: n.a.
  \item
    Responsible: Requirements Manager, Implementation Leader
  \end{itemize}
\end{enumerate}

\subsection{Closeout plan}\label{closeout-plan}

Not of any importance for this project.

\section{Technical Process Plan}\label{technical-process-plan}

\subsection{Process model}\label{process-model}

We will be using the \hyperref[IterativeIncremental]{Iterative and
Incremental development model} with some ideas of
\hyperref[AgileDev]{Agile Software Development}, which is based on this
model. This method has been chosen firstly because of the agenda of the
project which consists of an incremental delivery based on four
iterations. Secondly, it has been chosen for its simplicity and added
value: we focus on a working application per iteration which can than be
discussed with the client. In this way we open ourselves up to
requirements changes which will be given to us by the client at the end
of each iteration. This results in a continuous delivery of valuable
software, one of the key principles of agile development. The figure
below shows \hyperref[SpiralModel]{(Boehm's) spiral model}, which will
be used as development process model.

\begin{itemize}
\itemsep1pt\parskip0pt\parsep0pt
\item
  One iteration consists of four phases:

  \begin{itemize}
  \itemsep1pt\parskip0pt\parsep0pt
  \item
    Determination of objectives: In this phase changes in requirements
    will be determined and introduced in the SRS\\
  \item
    Identification (and resolving) risks: This phase involves the
    identification of possible risks caused by changed requirements:
    these will be discussed in the weekly meetings
  \item
    Developing and testing: Using the changed/extended design of the
    previous iteration, new and changed functionality will be
    implemented by the developers. The coordination of this phase is
    being done by the implementation leader, software quality assurance
    manager, design manager and test manager
  \item
    Planning of the next iteration: Releasing the new version of the
    product is the first thing to be done in this iteration. After this,
    the next iteration will be planned by the project manager.
  \end{itemize}
\end{itemize}

\subsection{Methods, tools and
techniques}\label{methods-tools-and-techniques}

At the moment of writing, the programming language has not been chosen
yet and will therefore not be mentioned in this version of the SPMP.
Therefore, some items will be added to the list below in future versions
of the SPMP.

\begin{itemize}
\itemsep1pt\parskip0pt\parsep0pt
\item
  Github will be used as

  \begin{itemize}
  \itemsep1pt\parskip0pt\parsep0pt
  \item
    communication tool for documents
  \item
    versioning control system for source code
  \end{itemize}
\item
  Eclipse will be used as \hyperref[IDE]{IDE} during the implementation
  process.
\item
  A MySQL database will be used as backend database on Wilma. It will be
  populated with course schedule data.
\end{itemize}

\section{Supporting Process Plans}\label{supporting-process-plans}

\subsection{Software Configuration Management Plan
(SCMP)}\label{software-configuration-management-plan-scmp}

\subsubsection{Software Configuration Management
Plan}\label{software-configuration-management-plan}

\paragraph{Introduction}\label{introduction}

In this document we will describe the workflow that needs to be followed
for the Xiast project. We will also talk about the two \textbf{Git}
repositories that will be used. This document will also be included in
an adapted form in the SPMP.

\paragraph{The repositories}\label{the-repositories}

For this project we will be using two distinct Git repositories, each
with their own specific purpose.

\subparagraph{xiast-docs}\label{xiast-docs}

The \texttt{xiast-docs} repository will be used to hold all documents
related to the project. This includes reports made during meetings and
all plans made by the different leaders and managers.

Every manager has his own directory inside the \texttt{management}
directory. In this directory, the manager can put all files related to
his work and reports. If a manager wants to make a change to the
directory structure, for instance a new folder inside the management
directory, an issue (see below) should be created and closed by the
software configuration manager. By doing this the software configuration
manager can keep the directory structure in the SCMP up-to-date. Issues
should not be created for structural changes inside one's own directory.

The current directory structure, which may be subject to change, is:

\begin{itemize}
\itemsep1pt\parskip0pt\parsep0pt
\item
  \textbf{/}

  \begin{itemize}
  \itemsep1pt\parskip0pt\parsep0pt
  \item
    \textbf{management} directory containing all official documentation

    \begin{itemize}
    \itemsep1pt\parskip0pt\parsep0pt
    \item
      \textbf{configuration} concerning project configuration
    \item
      \textbf{design} concerning software design
    \item
      \textbf{implementation} concerning implementation
    \item
      \textbf{project} concerning project planning and management
    \item
      \textbf{quality} concerning software quality and testing
    \item
      \textbf{requirements} concerning all software requirements
    \end{itemize}
  \item
    \textbf{meetings} contains all agendas and reports of meetings
  \item
    \textbf{manuals} contains all files related to the Xiast manuals
  \end{itemize}
\end{itemize}

\subparagraph{xiast}\label{xiast}

The \texttt{xiast} repository is the main repository that will contain
all the code for both the server and the Xiast website. The current
directory structure, which again may be subject to change, is:

\begin{itemize}
\itemsep1pt\parskip0pt\parsep0pt
\item
  \textbf{/}
\end{itemize}

The \texttt{xiast} repository is for code and related resources
(graphics, sound, SQL, \ldots{}) only.

\paragraph{Tracking issues}\label{tracking-issues}

With GitHub's issue tracker, which can be found at\\
\texttt{https://github.com/se1-1314/xiast/issues}, we can create
so-called \emph{issues}. An issue can be anything from a bug, request
for implementation or suggestion, goal, \ldots{} By using this tool we
can achieve a workflow that will make it easier for both us and others
to track the progress of the project.

First of all, when all requirements of the project are known and we know
more about the design of the software, we will split up every
requirement in one or more smaller ``tasks'' that need to be finished in
order to implement the requirement. These issues can then be bundled
into a \emph{milestone} which denotes the requirement that needs to be
implemented. Milestones feature progress trackers which again makes it
really easy for people to see how much still needs to be done for a
requirement.

For example: ``Functioning user system'' can be a milestone, with issues
such as ``User registration'' and ``User login'', or ``Access control''
with ``Assigning user rights'' and ``Rights checking'' as some of the
issues.

Furthermore, if bugs are found during testing or actual use of the
application, issues can be made for these bugs which will be assigned to
a special milestone exclusively for bug fixing.

Requests for implementation and suggestions can also be made through the
creation of issues. If a request or suggestion gets denied, we can just
simply close the issue (issues can always be re-opened). If the request
or suggestion gets accepted, the issue needs to be assigned to a
milestone specific for this issue.

It is important that the issue tracker is kept clean at all times. What
this means is that all issues and milestones have proper descriptions
and descriptive titles, issues that are finished must be closed (see
workflow), when one team member takes over an issue from another team
member this must be updated on the tracker, \ldots{} and so on.

The usage of the issue tracker, when done properly, causes less overhead
on the project because we do not need to maintain our own tracker or
website.

\paragraph{Git workflow}\label{git-workflow}

When working with Git, there are multiple workflows possible.
Considiring the size of the team and project, we will be using the so
called \textbf{feature branching workflow}. This workflow allows us to
tightly integrate the issue tracker into our project.

\subparagraph{Branches}\label{branches}

There will always be one central branch: the \textbf{master} branch.
This branch is the actual ``master copy'' of our project and should
contain preferably only working code.

Whenever an issue needs to be implemented, it first needs to be assigned
to a team member who will then be responsible for the implementation of
said issue. The member will then locally, after pulling the repository
first, create a new branch from the \texttt{master} branch and name it
after the issue. He will then checkout the newly made branch, from now
on the working branch, and publish it to GitHub.

All work related to implementing the feature will then be done on the
working branch and that branch only. Regulary pushing commits made in
the working branch is a must.

\subparagraph{Merging}\label{merging}

After the work has been done and the issue is implemented, the working
branch needs to be merged into the \texttt{master} branch. It is mainly
the team member's responsibility to avoid merge conflicts!

Conflicts can be avoided more easily by periodically pulling and merging
the \texttt{master} branch into the working branch, and not vice versa.
By doing this regulary, conflicts that will occur will be smaller than
they would be if we didn't merge at all.

If and only if all work is done and the working branch is conflict free
and the head of the branch is pushed to GitHub, steps can be taken to
merge the working branch into the \texttt{master} branch. This is done
by making a \textbf{pull request}.

\subparagraph{Pull request}\label{pull-request}

When making a pull request, select the \texttt{master} branch as the
base branch and the working branch as the copmare branch. In the
description of a pull request a link to the appropriate issue of the
working branch should be included.

After making the pull request, it is the implementation manager's
responsibility to accept or deny the request. The first step is to
review the code by testing it locally to see if it actually works. If he
so chooses, he can also simulate the merging locally to see if it works
with previous accepted pull requests.

If the code works and there are no conflicts, the pull request can be
accepted and close. After this, the appropriate issue should be closed
too and a comment with a link to the pull request should be made. To
avoid ``branch pollution'' the working branch can safely be removed from
the repository by GitHub after accepting the pull request.

If conflicts arise during merging because of for instance earlier
accepted pull requests, the implementation manager can choose to fix the
conflicts himself manually or just denying and closing the pull request.
The second option is advised. In this case, the team member must fix his
code and make a new pull request. The same is valid for when there are
problems with the code itself.

\subparagraph{Reassigning issues}\label{reassigning-issues}

When a member is assigned to an issue and can't implement or solve it
because of reasons, the issue can be reassigned to another team member.
In this case, the previous member should make sure the latest commits on
the working branch are pushed to GitHub. The new member can then easily
pick up the previous member's work.

Updating the issue on the tracker must not be forgotten either!

\subparagraph{docs}\label{docs}

For the \texttt{xiast-docs} repository, we don't need to use the feature
branch workflow, we can just use a \textbf{centralized workflow}. This
means we won't use branching and we will just commit to the master
branch.

\subparagraph{Tagging}\label{tagging}

At the end of every iteration of the project, the project manager is
responsible to make an \emph{annotated tag} for both repositories. The
names of the tag should follow the following naming convention:
\texttt{iteration-\{i\}.\{e\}} where \texttt{\{i\}} is the current
iteration and \texttt{\{e\}} stand for last minute edits after making
the tag starting with 0. This leaves room for a small amount of error.

\paragraph{Project website}\label{project-website}

One of the requirements of the project is a website that can be used to
track the progress of the project. Instead of focussing on implementing
our own system, we will make full use of the tools GitHub is providing
us.

The website can be reached at \texttt{http://se1-1314.github.io/xiast}.
This page, which is generated using GitHub Pages, contains information
about the project, including download links for the source code and
links to various documents.

GitHub Pages is in essence a static website generator. This means that,
given the contents of the site, it only needs to generate all the HTML
and related files once. This not only makes it faster, but also a lot
easier.

The automatic page generator can be found on the settings page of the
project. The files it generates are located in the \texttt{gh-pages}
branch of the \texttt{xiast} repository. Commiting to this branch thus
allows us to edit the page manually, although it is highly advised to
keep using the automatic page generator and edit the content of the site
using markdown.

To keep track of the project's status however, we will use the built-in
issue tracker of GitHub.

\subsection{Verification and Validation Plan
(STD)}\label{verification-and-validation-plan-std}

This plan will be delivered Friday 15th 2013, 2013: deadline for the
other documents.

\subsection{SoftwareDocumentation Plan
(SDP)}\label{softwaredocumentation-plan-sdp}

This plan will be delivered Friday 15th, 2013: deadline for the other
documents.

\subsection{Software Quality Assurance Plan
(SQAP)}\label{software-quality-assurance-plan-sqap}

No seperate plan required for this project. All relevant information
concerning this plan can be found in this document, the SPMP. The QAM is
responsible for this.

\subsection{Problem Resolution Plan}\label{problem-resolution-plan}

This plan will be delivered Friday 15th, 2013: deadline for the other
documents.

\section{Additional Plans}\label{additional-plans}

Following documents play also a role of importance in this project: SRS,
SDD They will be delivered Friday 15th, 2013: deadline for the other
documents.

\end{document}