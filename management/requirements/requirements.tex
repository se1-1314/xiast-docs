\documentclass[9pt]{article} 

\usepackage[utf8]{inputenc}
\usepackage{geometry}
\usepackage{amsmath}
\usepackage{mathtools}
\usepackage{fixltx2e}
\usepackage{url}
\usepackage{graphicx}
\usepackage{mathabx}
\usepackage{float}
\usepackage{subfig}
\usepackage{caption}
\usepackage{picture}
\usepackage{verbatim}
\usepackage{pdfpages} 

\geometry{a4paper} 
\geometry{margin=3cm} 

\title{Software Requirements Specifications\\Version 1.0}
\author{Anders Deliens\\Software Engineering 2013-2014 Group 1}
% \date{} 

\begin{document}
\maketitle

\section{Introduction}\label{introduction}

\subsection{Purpose}\label{purpose}

The purpose of this SRS is to provide a detailed description of all the
requirements for project Xiast. It will include a list of constraints,
features and (user)interfaces.

This document is intended for the members of the development team behind
Xiast as well as Professor Ragnhild Van Der Straeten and assistant Jens
Nicolay.

Disclaimer: This document is a work in progress and is not yet complete.
If you have any questions or suggestions, please let me know.

\subsection{Scope}\label{scope}

The application that will be discussed in this document is an online
scheduling-tool for university classes, named Xiast (short for Xiast is
a scheduling tool). It will compute the optimal distribution of classes,
based on certain constraints, which are given by administrators and/or
teachers. Students and teachers will be able to check their personal
schedule online and via android smartphones. Teachers will also be able
to add certain requirements to the classes and will be able to change
some of these requirements or even cancel a class last-minute. The goal
of this project is to make a scheduling-tool that is very personally
modifiable and user-friendly.

\subsection{Definitions, acronyms and
abbreviations}\label{definitions-acronyms-and-abbreviations}

\begin{itemize}
\itemsep1pt\parskip0pt\parsep0pt
\item
  program-administrator = person who is able to configure every detail
  of the scheduling constraints: courses, classrooms, teacher, number of
  students, theory or practical classes\ldots{}
\item
  teacher = person who defines the details of all the classes assigned
  to him: assistants,required resources and facilities,
  assistants\ldots{}
\item
  student = someone who can register for programs and courses
\item
  user = program administrator, teacher or student
\item
  guest = someone who is not yet logged in and identified as user
\item
  program = combination of multiple courses, as provided by the
  university
\item
  WPO = lab sessions or exercise lessons
\item
  HOC = theory lessons
\end{itemize}

\subsection{References}\label{references}

IEEE Std 830-1998, IEEE Recommended Practice for Software Requirements
Specifications, \emph{IEEE Computer Society}, 1998

\subsection{Overview}\label{overview}

Section 2 will cover an overview of the general requirements. Using
scenarios it will describe how the application should work. The
functions that will be implemented and the user characteristics will be
portrayed there, next to the constraints the application is bound by.
The requirements will be further analysed and described in-depth in
section 3.

This document largely follows the IEEE Std 830-1998.

\section{Overall description}\label{overall-description}

\subsection{Product perspective}\label{product-perspective}

Xiast is an independent open-source product and will be free to use by
anyone. Xiast aims to provide more personally modifiable rosters than
other applications on the market, using a user-friendly interface.
Alongside the web-application, there will be a mobile android
application with some specific features and support for mobile users.
Both the web and mobile application will use data stored on a
database-server (Wilma).

\subsection{Product functions}\label{product-functions}

Xiast will accommodate three types of users: students, teachers and
program-administrators. Each of these types has their own rights and own
functions.

Every user will have acces to the home screen, which will contain
information about Xiast, the team, the VUB\ldots{} as well as an option
menu where the user can choose his preferred language. There will be a
button the user can click to go the log-in screen. Here the user will
have to type in his username and password. Xiast will check the
authenticity of the user and will determine wether the user is a
student, teacher or program-administrator.

Students can register (and unregister) for a program or different
courses. After a certain amount of time, no further registration will be
allowed. The students will be able to check their personal schedule
online and on their smartphone. They will be notified of any last-minute
changes made to their schedule.

Teachers get certain courses assigned by program-administrators. They
will be able to check their roster online and on their smartphone. They
will be able to send their specifications regarding their courses (like
maximum number of students, or the need of a overhead-projector, days
they won't be able to teach\ldots{}) to the program-administrator, who
will try to implement these request in the scheduling, if possible. A
teacher can do a scheduling for his own courses, and can mark this as a
good roster for him, send it to a program-administrator, who will try to
incorporate this roster in the final roster. If a teacher, for some
reason, cannot make it to his scheduled class, he can notify the
application, which will update the rosters for all students attending
said class. Teachers can request certain changes to their personal
roster, but only a program administrator is able to actually change the
roster and the details of it.

Program administrators are in charge of a bundle of courses, which
combined together form a standard program. They specify which courses
are part of the program, how many courses their will be, which courses
are obligatory and which aren't\ldots{} They also assign a teacher to
each course. They can make or delete courses and are able to modify
every constraint regarding the scheduling.

The program administrator will input certain courses and constrains and
Xiast will compute and display the best roster possible given said
constraints. If there are any overlaps, Xiast will highlight them and
the program administrator has the choice to either make a change
manually or let Xiast try come up with alternative solutions. After any
(manual) changes are made, the administrator has to give his fiat before
the roster gets picked and made visible for all other users. When a new
schedule is being made and an overlap between courses occurs, Xiast will
try to adjust the scheduling of certain courses by ignoring their lowest
priority constraints. By ignoring these low priority constraints, a new
roster without overlaps may emerge. The administrator of the course
which has had a change in constraints or which needs to be rescheduled
will receive a notification. A program administrator may choose to
change the scheduling or constraint of the courses manually or even keep
the overlap.

\subsection{User Characteristics}\label{user-characteristics}

Users don't need any particular experience or expertise to use this
application, just some basic knowledge of how to work with a computer
and internet (on smartphones).

\subsection{Constraints}\label{constraints}

Since the application must be able to run on Android devices, there is a
limitation to memory. Xiast must be able to store its data in an
efficient way, so that it runs smoothly, even on devices with little
internal memory and low processor speeds. The schedules should still be
nicely displayed on smaller smartphone screens. If the mobile network
connection is unstable, the application should still be able to display
the schedule, without having to reload everything.

The application has to have a Wilma back-end and a browser (computer or
android smartphone) front-end

Only Java, JavaScript, HTML, CSS, SQL and associated open-source
frameworks and libraries may be used as programming language. Only free
and open-source software may be used for this product.

The application has to be easy to install and the user interface must be
appealing and simple.

\subsection{Assumptions and
dependencies}\label{assumptions-and-dependencies}

To use this application, an internet connection is required. This
connection is necessary for the application to fetch data from the
server. If a user doesn't have access to internet, he will still be able
to check previously saved schedules.

\section{Specific Requirements}\label{specific-requirements}

This section will be updated in the future with flow-charts, graphics
and mock-ups to clarify certain requirements, as well as improve the
readability.

\subsection{External interfaces}\label{external-interfaces}

The only link with an external system is the one to the Wilma back-end
server, which contains all the scheduling data.

\subsection{Functional
Requirements}\label{functional-requirements}

This next section will contain a detailed list of all functional
requirements that need to be implemented, divided into sections for
every user class. Every requirement will have a unique ID, which makes
it easier to make references. The ID consist of FR (=Functional
Requirement) followed by the first letter of the user class
(G=Guest;U=User;S=Student;T=Teacher and P=Program-administrator) and
then a number.

\paragraph{User Class 1: Guest}\label{user-class-1-guest}

\subparagraph{Functional Requirement
1.1}\label{functional-requirement-1.1}

\textbf{ID}: FRG1\\\textbf{TITLE}: Log in\\\textbf{DESCRIPTION}: When a
guest correctly enters the username and password of an existing account,
he will be logged in, giving him the appropriate
rights.\\\textbf{PRECONDITION}: ``Home'' screen is
showing.\\\textbf{SCENARIO}: A guest clicks on the log in tab. The log
in screen is showing. The guest enters his username and password and
clicks the ``log in'' button.\\The system logs the guest in as user and
gives him/her the appropriate rights and shows the appropriate
``Account'' screen.\\\textbf{EXCEPTIONS}: At any point in this scenario,
the guest can click on any of the other tabs on the home screen. In this
case, the guest will be redirected to the appropriate screen. If a guest
enters the wrong password or username, the system will display an error
message stating that either the password was incorrect or the username
doesn't exist. The guest will be than be redirected to the log in
screen, where he can try to log in again.\\\textbf{POSTCONDITION}: The
guest is now logged in as a user and the system is showing his/her
``Account'' screen.

\subparagraph{Functional Requirement
1.2}\label{functional-requirement-1.2}

\textbf{ID}: FRG2 (see also FRU2)\\\textbf{TITLE}: Cycling
tabs\\\textbf{DESCRIPTION}: Guests can cycle through the tabs on the
home screen.\\\textbf{PRECONDITION}: ``Home'' screen is
showing\\\textbf{SCENARIO}: Guests can click on any of the tabs on top
of the home screen and the system will direct them to the correct
page.\\\textbf{EXCEPTIONS}: None\\\textbf{POSTCONDITION}: The
appropriate screen is showing.

\subparagraph{Functional Requirement
1.3}\label{functional-requirement-1.3}

\textbf{ID}: FRG3\\\textbf{TITLE}: Choosing
language\\\textbf{DESCRIPTION}: Guests can choose their language by
clicking on the language tab.\\\textbf{PRECONDITION}: Home screen is
showing\\\textbf{SCENARIO}: A guest clicks on the language tab. A list
of all available language will be shown to the guest and he will be able
to choose his preferred language.\\\textbf{EXCEPTIONS}: If the guest
clicks on another tab before choosing the preferred language, he will be
redirected to the appropriate page, without any changes to the
language.\\\textbf{POSTCONDITION}: The guest is returned to the screen
he was using before clicking on the language tab, and this screen is now
translated to the preferred language. All the other pages of the
application will also be translated to the preferred language.

\paragraph{User Class 2: User}\label{user-class-2-user}

All the functional requirements that follow, will apply to all three
types of users: students, teachers and program-administrators.

\subparagraph{Functional Requirement
2.1}\label{functional-requirement-2.1}

\textbf{ID}: FRU1\\\textbf{TITLE}: Log out\\\textbf{DESCRIPTION}: Users
can log out, becoming a guest.\\\textbf{PRECONDITION}: User is logged
in.\\\textbf{SCENARIO}: Whenever a user clicks the log out button, he
will be asked to confirm this decision and after confirmation he will be
logged out. He will be returned to the home screen as a
guest.\\\textbf{EXCEPTIONS}: If the guest clicks cancel after clicking
the log out button, he stays logged in, remaining on the same page he
was on before clicking the log out button. If the user clicks another
tab after clicking the log out button, the system will interpret this as
a cancel log out command.\\\textbf{POSTCONDITION}: The user is logged
out and is now a guest.

\subparagraph{Functional Requirement
2.2}\label{functional-requirement-2.2}

\textbf{ID}: FRU2 (see also FRG2)\\\textbf{TITLE}: Cycling
tabs\\\textbf{DESCRIPTION}: Users can cycle through the tabs on the home
screen.\\\textbf{PRECONDITION}: ``Home'' screen is
showing\\\textbf{SCENARIO}: Users can click on any of the tabs on top of
the home screen and the system will direct them to the correct
page.\\\textbf{EXCEPTIONS}: None\\\textbf{POSTCONDITION}: The
appropriate screen is showing.

\subparagraph{Functional Requirement
2.3}\label{functional-requirement-2.3}

\textbf{ID}: FRU3\\\textbf{TITLE}: Viewing
schedule.\\\textbf{DESCRIPTION}: Users can view their personal schedule
and filter the display of the schedule with different
modifiers.\\\textbf{PRECONDITION}: None\\\textbf{SCENARIO}: Users click
on the view button and will be redirected to the view page. Here they
will be able to see their schedule. They can denote the month, week or
day they want to see as well as which program or course they want to
view. They can also choose to see the schedule of one course or of one
classroom.\\\textbf{EXCEPTIONS}: When there is no schedule to be shown,
a message will inform the users. When no specifications are made for the
time to be displayed, the current week will be
shown.\\\textbf{POSTCONDITION}: The correct personal schedule is being
displayed.

\paragraph{User Class 3: Student}\label{user-class-3-student}

\subparagraph{Functional Requirement
3.1}\label{functional-requirement-3.1}

\textbf{ID}: FRS1\\\textbf{TITLE}: View courses\\\textbf{DESCRIPTION}:
Students can take a look at all the courses they
follow.\\\textbf{PRECONDITION}: None\\\textbf{SCENARIO}: When a student
clicks on the ``Courses'' button, he will be redirected to a page where
a list of all his registered courses will be
displayed.\\\textbf{EXCEPTIONS}: If the student should not have any
registered courses, a message will notify the
student.\\\textbf{POSTCONDITION}: The correct list of registered courses
of the student is displayed.

\subparagraph{Functional Requirement
3.2}\label{functional-requirement-3.2}

\textbf{ID}: FRS2\\\textbf{TITLE}: Search for course or
program\\\textbf{DESCRIPTION}: Students can search a course or
program\\\textbf{PRECONDITION}: Students must be on the ``Courses'' page
in order to perform a search.\\\textbf{SCENARIO}: The student can type
in a query inside the search box on the ``courses'' screen and a list of
all courses/programs matching the query will be displayed. There will be
an option for the student to limit his search to courses from certain
programs, or years only.\\\textbf{EXCEPTIONS}: When there are no
courses/programs that match the search criteria, a notification will
alert the student.\\\textbf{POSTCONDITION}: A list matching the search
criteria will be displayed.

\subparagraph{Functional Requirement
3.3}\label{functional-requirement-3.3}

\textbf{ID}: FRS3\\\textbf{TITLE}: Registering for a course or
program\\\textbf{DESCRIPTION}: Students can register for courses and
programs\\\textbf{PRECONDITION}: Students must be on the ``Courses''
page in order to register for courses or programs.\\\textbf{SCENARIO}:
If a student wants to register for a course, he searches this course
using the search box. When the course is found, the student can click a
``plus''-symbol next to the course. A pop-up will ask for confirmation
and after the confirmation the student will be
registered.\\\textbf{EXCEPTIONS}: When there are no courses that match
the search criteria, a notification will alert the student. If a student
presses cancel he will not be registered for the
course.\\\textbf{POSTCONDITION}: The student is now registered for the
course/program.

\subparagraph{Functional Requirement
3.4}\label{functional-requirement-3.4}

\textbf{ID}: FRS4\\\textbf{TITLE}: Unregister for a course or
program\\\textbf{DESCRIPTION}: Students can unregister for courses and
programs.\\\textbf{PRECONDITION}: Students must be on the ``Courses''
page in order to register for courses or programs and must be registered
to at least one course.\\\textbf{SCENARIO}: If a student wants to
unregister for a course, he searches this course in the list of his
registered courses on his ``courses'' screen. When the course is found,
the student can click a ``minus''-symbol next to the course. A pop-up
will ask for confirmation and after the confirmation the student will be
unregistered.\\\textbf{EXCEPTIONS}: When a student presses cancel he
will not be unregistered.\\\textbf{POSTCONDITION}: The student is no
longer registered for the course/program.

\subparagraph{Functional Requirement
3.5}\label{functional-requirement-3.5}

\textbf{ID}: FRS5\\\textbf{TITLE}: Viewing course
details\\\textbf{DESCRIPTION}: Students can view the details of
courses.\\\textbf{PRECONDITION}: Students must be on the ``Courses''
page in order to view the details of a course.\\\textbf{SCENARIO}: If a
student wants to view the details of a course, he searches the course in
the list of his registered courses on his ``courses'' screen or by using
the search box. When the course is found, the student can click on the
course and the description of the course will be displayed. By clicking
the close button, the student can return to the ``courses''
screen.\\\textbf{EXCEPTIONS}: When the description of the course is
empty, a message will notify the student.\\\textbf{POSTCONDITION}: The
student is viewing the course details.

\paragraph{User Class 4: Teacher}\label{user-class-4-teacher}

\subparagraph{Functional Requirement
4.1}\label{functional-requirement-4.1}

\textbf{ID}: FRT1\\\textbf{TITLE}: View courses\\\textbf{DESCRIPTION}:
Teachers can take a look at all the courses that are assigned to
them.\\\textbf{PRECONDITION}: None\\\textbf{SCENARIO}: When a teacher
clicks on the ``Courses'' button, he will be redirected to a page where
a list of the courses that are assigned to him, will be
displayed.\\\textbf{EXCEPTIONS}: If the are no courses assigned to the
teacher, a message will notify him of that.\\\textbf{POSTCONDITION}: The
correct list of all assigned courses will be displayed.

\subparagraph{Functional Requirement
4.2}\label{functional-requirement-4.2}

\textbf{ID}: FRT2\\\textbf{TITLE}: Last-minute
cancelling\\\textbf{DESCRIPTION}: Teachers can cancel a scheduled course
last-minute.\\\textbf{PRECONDITION}: A course must be scheduled and the
teacher must go to his ``view'' screen.\\\textbf{SCENARIO}: When a
teacher can't make it to a certain class due to unexpected
circumstances, he can cancel that course by going to the ``view''
screen. He searches for the course that needs to be cancelled in the
schedule and clicks it. A pop-up will appear asking the teacher to
confirm cancellation or to go back without cancelling. This cancellation
is immediately visible for all the students attending this
course.\\\textbf{EXCEPTIONS}: When the teacher aborts the cancellation,
nothing happens and he is returned to the ``view''
screen.\\\textbf{POSTCONDITION}: The course is no longer scheduled and
this change is visible in all schedules containing this course.

\subparagraph{Functional Requirement
4.3}\label{functional-requirement-4.3}

\textbf{ID}: FRT3\\\textbf{TITLE}: Edit details of
course.\\\textbf{DESCRIPTION}: Teachers can change the details of a
course.\\\textbf{PRECONDITION}: A course must be assigned to the
teacher.\\\textbf{SCENARIO}: The teacher goes to his ``courses'' screen
and can click on any of the courses assigned to him. This will bring
forth another screen with different sections. One of these sections will
allow the teacher to adjust the description of the course (which can be
read by the students, see FRS5). Another section will contain a list of
possible facilities needed for the course. The teacher selects whether
he wants to adjust the facilities of the WPO or the HOC (since these can
differ quite a lot). Another section will contain his preferences for
how this course should be scheduled: which days, classrooms.. he
prefers. After making changes to either the description, the facilities
or his preferences, the teacher has to hit a save
button.\\\textbf{EXCEPTIONS}: The teacher can hit the cancel adjustments
button, so no changes will occur.\\\textbf{POSTCONDITION}: The course
description, facilities and preferences are changed and updated
correctly.

\subparagraph{Functional Requirement
4.4}\label{functional-requirement-4.4}

\textbf{ID}: FRT4\\\textbf{TITLE}: Teacher
auto-scheduling\\\textbf{DESCRIPTION}: Teachers can order a scheduling
of their own courses and mark one as a
``proposal''.\\\textbf{PRECONDITION}: Courses must be assigned to the
teacher.\\\textbf{SCENARIO}: The teacher can go to his ``view'' screen,
where his current schedule will be shown. On top of the screen there
will be a ``schedule'' button. When the teacher presses this button,
Xiast will perform a scheduling of all the courses assigned to the
teacher, taking the preferences and facilities of each course into
account. The new roster will be displayed and the teacher can cycle
between his current schedule and the new schedule by clicking on either
the ``current'' button or the ``schedule'' button. If the scheduling
doesn't contain any conflicts or overlaps, the teacher can mark this
schedule as a ``non-conflicting proposal''.\\\textbf{EXCEPTIONS}: If the
newly scheduled roster contains overlaps, the teacher will not be able
to mark it as a ``non-conflicting proposal''. Instead, it will be marked
with ``conflicting proposal'' so that it remains clearly visible that
there are still conflicts in this schedule. The conflicts will be
highlighted in order to make them more visible.\\\textbf{POSTCONDITION}:
A newly scheduled roster is marked as ``(non-)conflicting proposal'' and
this proposal is visible for the program-administrator.

\subparagraph{Functional Requirement
4.5}\label{functional-requirement-4.5}

\textbf{ID}: FRT5\\\textbf{TITLE}: Teacher manual
adjustments\\\textbf{DESCRIPTION}: Teachers can manually adjust their
schedule.\\\textbf{PRECONDITION}: Courses must be assigned to the
teacher. The teacher has done a scheduling.\\\textbf{SCENARIO}: The
teacher can change the times when the courses are being scheduled in
order to remove certain existing overlaps. If conflicts are resolved,
the courses lose their highlighted state. The teacher can mark the
schedule as ``non-conflicting proposal''.\\\textbf{EXCEPTIONS}:
Conflicts occur, the teacher cannot mark the schedule as
``non-conflicting proposal''. Instead, it will be marked with
``conflicting proposal''.\\\textbf{POSTCONDITION}: Manual changes have
been made and the new roster is marked as a ``(non-)conflicting
proposal''.

\paragraph{User Class 5:
Program-administrator}\label{user-class-5-program-administrator}

\subparagraph{Functional Requirement
5.1}\label{functional-requirement-5.1}

\textbf{ID}: FRP1\\\textbf{TITLE}: View programs and
courses\\\textbf{DESCRIPTION}: Program-administrators can look at all
the programs that they have to control.\\\textbf{PRECONDITION}:
None\\\textbf{SCENARIO}: When a program-administrator clicks on the
``Programs'' button, he will be redirected to a page where a list of the
programs that he is in control of, will be displayed. If he presses on
one of the programs, a list containing all the courses of that program
will be displayed.\\\textbf{EXCEPTIONS}: If the are no programs or if
there are no courses in a program, a message will notify the
program-administrator.\\\textbf{POSTCONDITION}: The correct list of all
the programs under control of the program-administrator will be
displayed.

\subparagraph{Functional Requirement
5.2}\label{functional-requirement-5.2}

\textbf{ID}: FRP2\\\textbf{TITLE}: Add Program\\\textbf{DESCRIPTION}:
Program-administrators can add a program to their
repertoire.\\\textbf{PRECONDITION}: None\\\textbf{SCENARIO}: When a
program-administrator wants to add a program, he goes to his
``Programs'' view and presses on the ``plus''-symbol. A pop-up window
shows up, where he has to fill in a name for the program. After hitting
the save button, the program is created.\\\textbf{EXCEPTIONS}: When the
program-administrator presses cancel, no program is created and he is
redirected to the ``Programs'' view.\\\textbf{POSTCONDITION}: A new
program is created.

\subparagraph{Functional Requirement
5.3}\label{functional-requirement-5.3}

\textbf{ID}: FRP3\\\textbf{TITLE}: Add Course\\\textbf{DESCRIPTION}:
Program-administrators can add a course to one of their
programs.\\\textbf{PRECONDITION}: The program-administrator should have
at least one program under his control.\\\textbf{SCENARIO}: When a
program-administrator wants to add a course, he goes to his ``Programs''
view and presses on the program where he wants to add the course. Then
he presses on the ``plus''- symbol. A form appears on the screen, where
he can fill in the name of the course, assign a teacher to it, state how
many hours the course has, note if its an obligatory course or an
optional one, define how many hours are WPO and how many are HOC. After
hitting the save button, the course is added to the
program.\\\textbf{EXCEPTIONS}: When the program-administrator presses
cancel, the course isn't added and he returns to the ``program''
screen.\\\textbf{POSTCONDITION}: The new course is added to a program.

\subparagraph{Functional Requirement
5.4}\label{functional-requirement-5.4}

\textbf{ID}: FRP4\\\textbf{TITLE}: Edit
program/course\\\textbf{DESCRIPTION}: Program-administrators can edit
the courses inside their programs.\\\textbf{PRECONDITION}: The
program-administrator should have at least one program under his
control, containing a minimum of one course.\\\textbf{SCENARIO}: The
program-administrator goes to his ``Programs'' screen and clicks on the
program he wants to edit. The list of courses of this program is shown.
The program-administrator can click on the ``minus''-symbol to remove
the course from the program, or he can click on the course. He then has
the option to click on ``edit course'' to change every detail of the
course. After hitting save, the form is updated.\\\textbf{EXCEPTIONS}:
When the program-administrator presses cancel, the course isn't changed
and he returns to the ``program'' screen. \textbf{POSTCONDITION}: Course
details are changed and/or courses are removed from a program.

\subparagraph{Functional Requirement
5.5}\label{functional-requirement-5.5}

\textbf{ID}: FRP5\\\textbf{TITLE}: View preferences/facilities
courses\\\textbf{DESCRIPTION}: Program-administrators can see the
preferences and facilities required for each course in their
program.\\\textbf{PRECONDITION}: The program-administrator should have
at least one program under his control, containing a minimum of one
course.\\\textbf{SCENARIO}: The program-administrator goes to the
``Programs'' screen and clicks on the program which contains the course
of which he wants to see the details of. He then presses on the course
and then hits the option ``view preference/facilities''. A pop-up window
shows the preferences and facilities put forward by the teacher of the
course. By hitting back, the program-administrator can return to his
``Programs'' view.\\\textbf{EXCEPTIONS}: When the teacher of the course
hasn't filled in the preferences and facilities of the course, the
program-administrator will be alerted with a
notification.\\\textbf{POSTCONDITION}: Course preferences and facilities
are shown on the screen.

\subparagraph{Functional Requirement
5.6}\label{functional-requirement-5.6}

\textbf{ID}: FRP6\\\textbf{TITLE}:
Auto-scheduling\\\textbf{DESCRIPTION}: Program-administrators can
command Xiast to perform a scheduling on the courses of their programs
and mark it as ``final''.\\\textbf{PRECONDITION}: The
program-administrator should have at least one program under his
control, containing a minimum of one course.\\\textbf{SCENARIO}: The
program-administrator goes to his ``Programs'' screen and clicks on the
program he wants to schedule. He clicks on ``Schedule'' and he is
redirected to the ``Schedule'' view. Xiast shows the best schedule it
could compute. If the program-administrator approves of this schedule he
can mark this schedule as ``final'' (or as ``non-conflicting
proposal''). Now this schedule will become visible for every
user.\\\textbf{EXCEPTIONS}: If the newly scheduled roster contains
overlaps, the schedule cannot be marked as a ``non-conflicting
proposal''. Instead, it will be marked with ``conflicting proposal'' so
that it remains clearly visible that there are still conflicts in this
schedule. The conflicts will be highlighted in order to make them more
visible.\\\textbf{POSTCONDITION}: The new ``final'' schedule is
available for everyone or the newly scheduled roster is marked as
``non-conflicting proposal''.

\subparagraph{Functional Requirement
5.7}\label{functional-requirement-5.7}

\textbf{ID}: FRP7\\\textbf{TITLE}: Manual
scheduling\\\textbf{DESCRIPTION}: Program-administrators can manually
perform changes to a schedule.\\\textbf{PRECONDITION}: The
program-administrator should have at least one program under his
control, containing a minimum of one course.\\\textbf{SCENARIO}: The
program-administrator can change the times when the courses are being
scheduled in order to remove certain existing overlaps. If conflicts are
resolved, the courses lose their highlighted state. He can mark the
schedule as ``non-conflicting proposal'' or as
``final''.\\\textbf{EXCEPTIONS}: Conflicts occur, the schedule cannot be
marked as ``non-conflicting proposal''. Instead, it will be marked as
``conflicting proposal''.\\\textbf{POSTCONDITION}: Manual changes have
been made and the new roster is marked as a ``(non-)conflicting
proposal''.The program-administrator goes to his ``Programs'' screen and
clicks on the program he wants to schedule. He clicks on ``Schedule''
and he is redirected to the ``Schedule'' view. Xiast shows the best
schedule it could compute. If the program-administrator approves of this
schedule he can mark this schedule as ``final''. Now this schedule will
become visible for every user.\\\textbf{POSTCONDITION}: The new
``final'' schedule is available for everyone or the newly scheduled
roster is marked as ``non-conflicting proposal''.

\subsection{Performance
requirements}\label{performance-requirements}

No particular performance related objectives were put forward for this
project, however the aim of this project is to make Xiast as performant
as possible. The number of students should not be of any concern, since
students don't communicate with each other. The number of teachers or
program-administrator also doesn't influence the performance of the
application, however time to negotiate between different teachers and
program-administrators to come up with a solution for overlaps will
increase, since they have to reach a consensus about the subject.

\subsection{Design constraints}\label{design-constraints}

There are no design constraints except that the application must be able
to run on android devices and only free, open-source
software/programming languages may be used.

\subsection{Software system
attributes}\label{software-system-attributes}

\paragraph{Reliability}\label{reliability}

The application should work the way it is intended to work at all time.
We aim not to have more than 50 errors the first year after launch and
the number of errors should decrease over the years.

\paragraph{Availability}\label{availability}

The application is available for the user as long as it is installed on
their smartphone or as long as they have access to a web-browser. Since
this projects works with an external server (Wilma), availability of the
database cannot be regulated by any of the team-members. However, the
goal is to have as less down-time as possible. Should there be a server
failure, the team will contact the person responsible for the server and
try to help get it back up as soon as possible.

\paragraph{Security}\label{security}

The database will be protected from SQL-injections.

\paragraph{Portability}\label{portability}

The application will be available for android devices and any
web-browser.

\subsection{Other requirements}\label{other-requirements}

The application should at least be available in Dutch and English.
\end{document}
